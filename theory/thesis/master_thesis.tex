\documentclass[english]{article}
\usepackage[utf8]{inputenc}
%\usepackage{babel}
\usepackage[backend=bibtex]{biblatex}
\usepackage{hyperref}
\usepackage{mathtools}
\usepackage[printonlyused]{acronym}
\usepackage{minted}


\hypersetup{
    colorlinks,
    citecolor=black,
    filecolor=black,
    linkcolor=black,
    urlcolor=black
}

\bibliography{master.bib}

% Add mathematical foo
\newcommand{\EI}{\operatorname{EI}}
\newcommand{\normal}{\mathcal{N}}

\begin{document}

\title{Bayesian Hyperparameter-Optimization in Grammer-Based Search Spaces}
\author{Moritz Meier}
\maketitle
\tableofcontents
\newpage

\section*{List of Acronyms}
\begin{acronym}
\acro{AI}{Artificial Intelligence}
\acro{ML}{Machine Learning}
\acro{BOA}{Bayesian Optimization Algorithm}
\acro{DAG}{Directed Acyclic Graph}
\acro{SMBO}{Sequential Model-Based Optimization}
\acro{SMAC}{Sequential Model-Based Algorithm Configuration}
\end{acronym}

\section{Introduction}
 - mathematical programming

\paragraph{Motivation}

\section{Global Black-Box Optimization}
 - Local Optimization
 - Global Optimization
 - Criteria for optimization

\subsection{Active Learning}


\subsection{Bandit Problem}



\section{Hyperparameter Optimization}

\subsection{Machine Learning as Optimization}

Most problems in \ac{AI} and \ac{ML} can be expressed) as
optimization problems.

 - Types and Error Functions
 - Model Selection

\begin{quote}
Machine learning algorithms, however, have certain characteristics that distinguish them from other black-box optimization problems.  First, each function evaluation can require a variable amount of time:  training a small neural network with 10 hidden units will take less time than a bigger net-work with 1000 hidden units.  Even without considering duration, the advent of cloud computing makes it possible to quantify economically the cost of requiring large-memory machines for learning, changing the actual cost in dollars of an experiment with a different number of hidden units.
Second, machine learning experiments are often run in parallel, on multiple cores or machines. In both situations, the standard sequential approach of GP optimization can be suboptimal.
\end{quote}
Copied from: \cite{snoek_practical_2012}

\subsection{Algorithm Types}
 - Manual Search
 - Random Search
 - Grid Search
\subsection{Search Space}
\subsubsection{Construction}
 - Categrorical
 - Discrete
 - Continunous

\subsection{Combinators}
Join and prod over parameter-lists form an idempotent semiring:

$(A, \cup)$ is a idempotent commutative monoid (or join-semilatice with zero) with identity element $\emptyset$:
$$(a \cup b) \cup c = a \cup (b \cup c)$$
$$\emptyset \cup a = a \cup \emptyset = a$$
$$a + b = b + a$$
$$a \cup a = a$$

$(A, \times)$ is a commutative monoid with identity element $[\ ]$:
$$(a \times b) \times c = a \times (b \times c)$$
$$[\ ] \cup a = a \cup [\ ] = a$$

Product left and right distributes over join:
$$a\times(b + c) = (a\times b) + (a\times c)$$
$$(a + b)\times c = (a\times c) + (b\times c)$$

Product by $\emptyset$ annihilates $A$:
$$\emptyset \times a = a \times \emptyset = \emptyset$$

\paragraph{Assumptions}
\subsection{Search Space Construction}
\subsubsection{Search Space Hierachy}


\begin{tabular}{ l | c | c }
Language Feature & Search Space & Parameter Structure \\
\hline
atomic parameters   & one-dim          & single value \\
join                & tree             &              \\
product / operators & multi-dim        & tree         \\
recursion           & infinite tree    &              \\
expression equality & DAG              &              \\
value identity      & ?                & DAG          \\
recursive values    & ?                & ?
\end{tabular}

\subsection{Demands to a good HPO-Algorithm}
A non-exaustive list of general criteria for a good hyperparameter optimization algorithm.

\section{Gaussian Processes Regression}
\subsection{Kernel Combination}
Kernel combination properties. Maybe: Why do they work!

\section{Bayesian Optimization}
What distinguished HPO from other non-HPO.

\subsection{Aquisition Functions}

\subsubsection{Probability of Improvement}
\subsubsection{Expected Information}

Expected Information (EI) is the most widely used aquistion function.

$$ \EI(y|x) \coloneqq \int_{-\infty}^{\infty} \max(0, y^*-y)p(y|x)dy $$

For the one-dimensional case a simple closes form formula can be derivated:

$$ \EI(y|x) = \int_{-\infty}^{y^*}(y^*-y)p(y|x)dy$$

$$ = \int_{-\infty}^{y^*}(y^*-y)\normal(y; \mu, \sigma)dy =
\int_{-\infty}^{y^*}(y^*-y)\normal(y-\mu; 0, \sigma)dy$$

$$ = \int_{-\infty}^{y^*}(y^*- t - \mu)\normal(t; 0, \sigma)dt $$

$$ = (y^*-\mu)\int_{-\infty}^{y^*}\normal(t; 0, \sigma)dt - \int_{-\infty}^{y^*}t\normal(t; 0, \sigma)dt$$

$$ = (y^*-\mu)\Phi(y^*/\sigma) - \phi(y^*/\sigma) $$

Where $\phi$ and $\Phi$ refere to the PDF and CDF of the standart normal distribution.

\subsubsection{Entropy search}
\subsubsection{Upper confidence bound}

\subsection{SMBO}
Sequential model-based optimization is a very general algorithmic pattern for global optimization with aquistion functions.
\subsection{SMAC}


\subsection{The Tree-Parzen Algorithm}
The Tree-Parzen Algorithm \cite{bergstra_algorithms_2011} is a Bayesian-Optimization algorithm for tree-structured parameter spaces.

\subsection{Tree and Graph Kernels/Metrics}


\section{The Tree Farm Algorithm}
Presenting my algorithm for Bayesian optimization of Grammar-Based search spaces.

\subsection{Extention to DAG-structured search spaces}
Identifying equivalent function trees with computer algebra systems.

\subsection{Parallelization}
Simple trick to extend algorithm to parrallel execution.

\subsection{Integration with Manual Search}
Manual guidance before and during the optimization process.

\subsubsection{Prior Information}
\subsubsection{Search Hints}
\subsection{Objective Function Approximation}


\section{Implementation}
\subsection{Search Space Construction DSL}
The search space construction system has atomic spaces that can
be combined to construct more complex spaces.

\subsubsection{Atomic Spaces}
Atomic spaces cannot (as the name suggest)

\paragraph{Categorical}
Categoriacal parameters have a finite unstructured domain of numerical or non-numerical values. This type is used for nominal scaled parameters like labels in a classification problem, basis-functions in linear regression or different activation functions in feedforward neural networks.

\paragraph{Continuous}
Continues parametes are bounded or un-bounded intervals over the real numbers. The simplest way to construct them is by slicing \mint{python}|treefarm.R| wich represents the real numbers.
\begin{minted}{python}
from treefarm import *

p1 = R[0:]

inter = Intervall(sub=0, sub=float('inf'))
p2 = to_space(inter)


\end{minted}


\paragraph{Discrete}



\subsubsection{Combinators}

\subsubsection{Functions}

\subsubsection{Operators}

\subsection{Function Graph Simplification}


\subsection{Algorithm Templates}


\section{Experiments}



\subsection{Gaussian Process Regression}
\subsubsection{Kernel Combination Optimization}
\subsubsection{Kernel Parameter and Kernel Combination Optimization}
\subsection{Elastic Net Linear Regression}
\subsection{Feat-Forward Networks}

\section{Conclusion}

\section{Outlook}
\paragraph{Connection to Functional Programming}
\paragraph{Connection to Constraint Satisfaction Problems}
\paragraph{Connection to Probabilistic Programming}
\paragraph{Connection to Reinforcement Learning}
\paragraph{Integration with Workflow-Managers}


\printbibliography

\end{document}
